\documentclass[12pt,twoside]{article}

%All course-specific variables are defined in a .cnf file for the course.

\newcommand{\InstructorName}{Yrd.Doç.Dr. Alper Yilmaz}
\newcommand{\Institution}{Yıldız Teknik Üniversitesi}
\newcommand{\CourseNumber}{BYM3532 }
\newcommand{\CourseTitle}{Viroloji Gr.1}
\newcommand{\OfficeHours}{emaille}
\newcommand{\Email}{alyilmaz@yildiz.edu.tr}
\newcommand{\Semester}{Bahar 2017}
\newcommand{\Website}{http://www.yarbis.yildiz.edu.tr/alyilmaz/course/viewCourse/id/6392}
\newcommand{\DateTime}{Cuma 9:30-12:00}
\newcommand{\Room}{KMB216}

\usepackage[english]{babel}
\usepackage{fontspec}

% Font choice:
%
% Garamond Premier Pro, unfortunately, costs money.
%
% Any owner of Adobe Reader actually has copies of the excellent, quite
% full-featured OpenType font Minion Pro. Look in Reader's application
% files.
%
% On a Mac, Hoefler Text makes a reasonable fallback.
%
% Finally, there are some free Palatino variants. For use with xelatex,
% I have been okay with TeX Gyre Pagella (an open-source font). If
% you wish, you could remove the xelatex/fontspec dependency here and
% instead use pdflatex with package mathpazo.

% original font spec
 \setmainfont[Ligatures=TeX,Numbers=OldStyle,%
    BoldFont={* Bold}]{Times New Roman}

\usepackage{xunicode}
\usepackage{xltxtra} 

\usepackage{hanging}
\usepackage{fancyhdr}
\usepackage{geometry}
\usepackage{setspace}
\usepackage{xkeyval}
\usepackage[bf,small,raggedright,compact]{titlesec}
\usepackage{enumerate}


%% pandoc complains about this so following url suggested the following command:
%% http://tex.stackexchange.com/questions/257418/error-tightlist-converting-md-file-into-pdf-using-pandoc
\providecommand{\tightlist}{%
  \setlength{\itemsep}{0pt}\setlength{\parskip}{0pt}}

% This file, with version control info, is generated by running ./vc 
% alper's note: my version is not integrated with git yet, so get static variables from vc2.tex
%%% This file has been generated by the vc bundle for TeX.
%%% Do not edit this file!
%%%
%%% Define Git specific macros.
\gdef\GITHash{1bea0e45b7f5c9ce496eb7ec58a8ac6967b98a80}%
\gdef\GITAbrHash{1bea0e4}%
\gdef\GITParentHashes{9db5ff6aa7410ac7cd278aaef21f8ba99699c558}%
\gdef\GITAbrParentHashes{9db5ff6}%
\gdef\GITAuthorName{Alper Yilmaz}%
\gdef\GITAuthorEmail{alperyilmaz@gmail.com}%
\gdef\GITAuthorDate{2016-02-24 03:29:55 +0200}%
\gdef\GITCommitterName{Alper Yilmaz}%
\gdef\GITCommitterEmail{alperyilmaz@gmail.com}%
\gdef\GITCommitterDate{2016-02-24 03:29:55 +0200}%
%%% Define generic version control macros.
\gdef\VCRevision{\GITAbrHash}%
\gdef\VCAuthor{\GITAuthorName}%
\gdef\VCDateRAW{2016-02-24}%
\gdef\VCDateISO{2016-02-24}%
\gdef\VCDateTEX{2016/02/24}%
\gdef\VCDateUSA{02/24/2016}%
\gdef\VCTime{03:29:55 +0200}%
\gdef\VCModifiedText{\textcolor{red}{with local modifications!}}%
%%% Assume clean working copy.
\gdef\VCModified{0}%
\gdef\VCRevisionMod{\VCRevision}%


% Some tweaks necessary to ensure annotations print at the end of 
% bibliography entries.
%%\usepackage{csquotes}
%%\usepackage[notes,annotation,short,backend=biber]{biblatex-chicago}
%%\DeclareFieldFormat{annotation}{#1\isdot}
%%\renewcommand{\bibfont}{\footnotesize}
%%\bibliography{course.bib}

\setcounter{secnumdepth}{-2}	% Suppress section numbers even with unstarred
				% (sub)section commands.

% adjust margins as you will
\geometry{hcentering=true,xetex}

% page headers. Set up for headers on odd-side pages only
\pagestyle{fancy}

\renewcommand{\footrulewidth}{0 pt} % I don't like fancyhdr's rules.
\renewcommand{\headrulewidth}{0 pt}
\fancyhead{}
\fancyhead[LO]{\small \CourseNumber{}}
\fancyhead[CO]{\small \Institution{}}
\fancyhead[RO]{\small \Semester{}}
\fancyfoot{}
\cfoot{\small \thepage}
% VCDateUSA macro supplied in tweaked vc-git.awk
\rfoot{\small Last revised \VCDateRAW}

% extra leading rather than indents to separate paragraphs
\singlespacing
\setlength{\parindent}{0 pt}
\setlength{\parskip}{0.25\baselineskip}

% overfull hboxes, begone
\setlength{\emergencystretch}{2 em}

\usepackage[dvipsnames]{xcolor}
\usepackage[
  pdftitle={\CourseTitle{}},
  pdfauthor=\InstructorName{},
  bookmarks, bookmarksopen,
  colorlinks=true,urlcolor=blue,citecolor=BlueViolet,
  xetex]{hyperref}
\urlstyle{same}

\begin{document}

% Sorry, world, but it's amazing how annoying it is to convince the
% titling package to do what I want with titles on documents like these.
%
% So no \maketitle.

\begin{flushleft}
\textbf{\CourseTitle{}} \\
\DateTime \ @ \Room \\
Instructor: \InstructorName \\
Email: \Email \\
Office Hours: \OfficeHours \\
\url{\Website} \\
\end{flushleft}


%%\section{Overview}\label{overview}

This course helps students to understand contemporary excitement and
fears about ``Big Data'' in a long historical context. Much is new about
the way corporations, governments, and individuals use massive
computational resources to search for patterns. But those who use big
data draw on legacies from well before the computer age for data
management, for structuring a complicated world into measurable
quantities.

\subsection{Course Themes}\label{course-themes}

We will trace the long history of big data through four parallel
strands:

\begin{enumerate}
\def\labelenumi{\arabic{enumi}.}
\item
  The rise of massive systems of data collection by the government in
  the 19th century through institutions like the census and the
  military.
\item
  The attempts of American businesses to collect and use data to control
  their markets and their workers.
\item
  The turn to data by the sciences.
\item
  The development of computers from the 1940s on, and the ways that
  social forces shaped the development of computing.
\end{enumerate}

\section{Ders Amaçları}\label{ders-amauxe7larux131}

Bu derste öğrenecekleriniz:

\begin{enumerate}
\def\labelenumi{\arabic{enumi}.}
\item
  Virüslerin karmaşık ve çeşitli olduğu.
\item
  Virüslerin hücresel mekanizmların anlaşılmasına yardımcı olduğunu ve
  bu mekanizmalarda ileri düzeyde gelişmiş oldukları.
\item
  Virüslerin hücredeki karmaşık sistemleri kendi yararlarına kullanarak
  ve bazen \textbf{dogmaları yıkarak} kısa yollardan karşılaştıkları
  sorunları aştıkları.
\end{enumerate}

%%\section{Requirements}\label{requirements}

\paragraph{Classroom}\label{classroom}

You must complete all the readings for the course and attend class
prepared to discuss them.

\paragraph{Response Posts to
Blackboard}\label{response-posts-to-blackboard}

6 times in the semester, you will post a short \emph{response} to one or
more of the readings for that day on Blackboard. These must be posted by
5pm the day \emph{before} class meets so that your peers have time to
read them. You must also write 6 \emph{responses} to your peer's posts.
Both posts and responses will be included as part of your participation
grade.

\paragraph{Papers}\label{papers}

You will write one 5 to 7 page paper for this class, based on the
readings; no outside research is expected.

\paragraph{Archival Project}\label{archival-project}

Mid-semester, we will take a trip to the university archives to look at
some archival documents. You will write up a description of another
document from elsewhere in the archives. This can take the form of a
straightforward narration, or you can adapt the information

\paragraph{Final Project}\label{final-project}

Final project assignments will be distributed in October, but you should
start thinking early about which one you will want. It will consist of
either 1) an 8-10 page paper in which you extend one of the weeks of the
course with additional readings; or 2) a digital project in which you
analyze a dataset created before the year 1994 using modern tools. In
either case, you must discuss the project in advance with me.

\section{Academic Integrity}\label{academic-integrity}

You are expected to have read, and follow at all times, the University's
\href{http://www.northeastern.edu/osccr/academicintegrity/index.html}{Academic
Integrity Policy}.

\section{Ders Materyalleri}\label{ders-materyalleri}

Derste kullanacağımız notlar \emph{Virology: Principles and
Applications} (John Carter and Venetia Saunders, 2007) adlı kitaptan
uyarlanmıştır. Bu kitaba ek olarak \emph{Fundamentals of Molecular
Virology} (Nicholas Acheson,2011) adlı kitaptan da yararlanabiliriz.

Ders notları fakülte karşısındaki fotokopi merkezinde basılı olarak,
\href{http://yarbis.yildiz.edu.tr/alyilmaz/course/viewCourse/id/6392}{YARBIS}
sayfasında da PDF formatında mevcuttur.

Ders notlarında maalesef çok resim ve az yazı bulunmaktadır. O yüzden
dersi dinleyip not almanız gerekmektedir.

\section{Notlandırma}\label{notlandux131rma}

Dönem sonu notları aşağıdaki dağılıma göre belirlenecektir:

\begin{itemize}
\itemsep1pt\parskip0pt\parsep0pt
\item
  Vize: 35\%
\item
  Final: 35\%
\item
  Quiz: 15\%
\item
  Ödev: 10\%
\item
  Katılım: 5\%
\end{itemize}

4 tane quiz yapılacak ve en yüksek 3 tanesi değerlendirmeye alınacaktır.
Bütün derslere katıldıysanız veya bir ders kaçırdıysanız katılım puanı
olarak 5 puan alacaksınız. Gelmediğiniz her 1-2 ders için bir puan
kaybedeceksiniz.

Ödevlerin sayısı ve içeriği ilerleyen derslerde duyurulacaktır. Ödev,
belirli birkaç virüsün biyomühendislik uygulamalarına dair olacaktır.
Ödevlerin sayısı ve daha detaylı içerikleri ilerleyen derslerde
duyurulacaktır.

Final sınavı vize sonrasında işlenen konuları içerir.

\section{İletişim}\label{iletiux15fim}

Emaillerime hızlı şekilde cevap vermeye çalışıyorum. 1-2 gün içinde
cevap alamazsanız lütfen hatırlatma emaili göndermekten çekinmeyiniz.

Sınav veya ödevlere dair tarih, saat ve içerik değişiklikleri
yapılabilir, fakat bu tür değişikliklerin sınıfta tartışarak
kararlaştırılması gerekmektedir, şahsi isteklere göre değişiklik
yapıldığında oldukça fazla iletişim sorunu yaşanmaktadır.

\section{Ders Planı}\label{ders-planux131}

Derslerin beklenenden uzun olması halinde aşağıdaki planda küçük
değişikler olabilir.

\subsubsection{16 Şubat. Giriş ve
Tanıtım}\label{ux15fubat.-giriux15f-ve-tanux131tux131m}

Viroloji ve ilgili alanların tanıtımı. Virüslerdeki karmaşıklığa
(complexity) ve çeşitliliğe genel bakış. Genom, protein açısından
virüslerin sahip olduğu farklılıklar. Kılıflı ve kılıfsız virüsler
arasında temel farklar.

\subsubsection{23 Şubat. Virolojide Kullanılan Metodlar ve Virus
Yapısı}\label{ux15fubat.-virolojide-kullanux131lan-metodlar-ve-virus-yapux131sux131}

Ders kitabının \emph{Chapter 2} ve \emph{Chapter 3} kısımları.

\subsection{Mekanizmlar}\label{mekanizmlar}

\subsubsection{1 Mart. Hücreye Bağlanma, Giriş. Translasyon ve
Taşıma}\label{mart.-huxfccreye-baux11flanma-giriux15f.-translasyon-ve-taux15fux131ma}

Ders kitabının \emph{Chapter 5} ve \emph{Chapter 6} kısımları.

\subsubsection{8 Mart. Virus Genom Replikasyonu, Assembly ve Çıkış,
Virus
Sınıflandırması}\label{mart.-virus-genom-replikasyonu-assembly-ve-uxe7ux131kux131ux15f-virus-sux131nux131flandux131rmasux131}

Ders kitabının \emph{Chapter 7} , \emph{Chapter 8} ve \emph{Chapter 10}
kısımları.

\subsection{Virüs Grupları}\label{viruxfcs-gruplarux131}

\subsubsection{15 Mart. Herpesvirüler ve Diğer dsDNA
Virüsler}\label{mart.-herpesviruxfcler-ve-diux11fer-dsdna-viruxfcsler}

Ders kitabının \emph{Chapter 11} kısmı.

\subsubsection{22 Mart. Parvovirüsler ve Diğer ssDNA
Virüsler}\label{mart.-parvoviruxfcsler-ve-diux11fer-ssdna-viruxfcsler}

Ders kitabının \emph{Chapter 12} kısmı.

\subsubsection{29 Mart. Reovirüsler ve Diğer dsRNA
Virüsler}\label{mart.-reoviruxfcsler-ve-diux11fer-dsrna-viruxfcsler}

Ders kitabının \emph{Chapter 13} kısmı.

\subsubsection{5 Nisan. Picornavirüsler ve Diğer Pozitif ssRNA
Virüsler}\label{nisan.-picornaviruxfcsler-ve-diux11fer-pozitif-ssrna-viruxfcsler}

Ders kitabının \emph{Chapter 14} kısmı.

\subsubsection{(12 Nisan). Vize}\label{nisan.-vize}

\subsubsection{19 Nisan. Rhabdovirüsler ve Diğer Negatif ssRNA
Virüsler}\label{nisan.-rhabdoviruxfcsler-ve-diux11fer-negatif-ssrna-viruxfcsler}

Ders kitabının \emph{Chapter 15} kısmı.

\subsubsection{26 Nisan. Retrovirüsler}\label{nisan.-retroviruxfcsler}

Ders kitabının \emph{Chapter 16} kısmı.

\subsubsection{3 Mayıs. Retrovirüsler (devam) \&
HIV}\label{mayux131s.-retroviruxfcsler-devam-hiv}

Ders kitabının \emph{Chapter 16} ve \emph{Chapter 17} kısımları.

\subsubsection{10 Mayıs. Hepadnavirüsler ve Diğer Reverse-Transcription
DNA
Virüsler}\label{mayux131s.-hepadnaviruxfcsler-ve-diux11fer-reverse-transcription-dna-viruxfcsler}

Ders kitabının \emph{Chapter 18} kısmı.

\subsubsection{17 Mayıs. Virüslerin Gelişimi ve Ortaya
Çıkışları}\label{mayux131s.-viruxfcslerin-geliux15fimi-ve-ortaya-uxe7ux131kux131ux15flarux131}

Ders kitabının \emph{Chapter 20} ve \emph{Chapter 21} kısımları.

\subsubsection{24 Mayıs. İnfektivite Direnci ve Aşılar (Son
Ders)}\label{mayux131s.-infektivite-direnci-ve-aux15fux131lar-son-ders}

Ders kitabının \emph{Chapter 23} ve \emph{Chapter 24} kısımları.


% Include all texts in bibliography database
\nocite{*}

% But skip any with keyword ``supplemental''
%% \printbibliography[notkeyword=supplemental,title=Full Citations for Readings]

\vfill
\footnotesize
\section{Acknowledgments}\label{acknowledgments}

This syllabus was adapted from
\href{https://github.com/bmschmidt/syllabus}{Benjamin Schmidt} and
\href{https://github.com/agoldst/tex}{Andrew Goldstone}.

This syllabus is available for duplication or modification for other
courses and non-commercial uses under a
\href{http://creativecommons.org/licenses/by-nc/3.0/}{CC BY-NC 3.0}
license. Acknowledgment with attribution is requested.

%%\input{input/gradingGuidelines.tex}

\end{document}


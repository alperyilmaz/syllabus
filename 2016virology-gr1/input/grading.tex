\section{Ders Materyalleri}\label{ders-materyalleri}

Derste kullanacağımız notlar \emph{Virology: Principles and
Applications} (John Carter and Venetia Saunders, 2007) adlı kitaptan
uyarlanmıştır. Bu kitaba ek olarak \emph{Fundamentals of Molecular
Virology} (Nicholas Acheson,2011) adlı kitaptan da yararlanabiliriz.

Ders notları fakülte karşısındaki fotokopi merkezinde basılı olarak,
\href{http://yarbis.yildiz.edu.tr/alyilmaz/course/viewCourse/id/6392}{YARBIS}
sayfasında da PDF formatında mevcuttur.

Ders notlarında maalesef çok resim ve az yazı bulunmaktadır. O yüzden
dersi dinleyip not almanız gerekmektedir.

\section{Notlandırma}\label{notlandux131rma}

Dönem sonu notları aşağıdaki dağılıma göre belirlenecektir:

\begin{itemize}
\itemsep1pt\parskip0pt\parsep0pt
\item
  Vize: 35\%
\item
  Final: 35\%
\item
  Quiz: 15\%
\item
  Ödev: 10\%
\item
  Katılım: 5\%
\end{itemize}

4 tane quiz yapılacak ve en yüksek 3 tanesi değerlendirmeye alınacaktır.
Bütün derslere katıldıysanız veya bir ders kaçırdıysanız katılım puanı
olarak 5 puan alacaksınız. Gelmediğiniz her 1-2 ders için bir puan
kaybedeceksiniz.

Ödevlerin sayısı ve içeriği ilerleyen derslerde duyurulacaktır. Ödev,
belirli birkaç virüsün biyomühendislik uygulamalarına dair olacaktır.
Ödevlerin sayısı ve daha detaylı içerikleri ilerleyen derslerde
duyurulacaktır.

Final sınavı vize sonrasında işlenen konuları içerir.

\section{İletişim}\label{iletiux15fim}

Emaillerime hızlı şekilde cevap vermeye çalışıyorum. 1-2 gün içinde
cevap alamazsanız lütfen hatırlatma emaili göndermekten çekinmeyiniz.

Sınav veya ödevlere dair tarih, saat ve içerik değişiklikleri
yapılabilir, fakat bu tür değişikliklerin sınıfta tartışarak
kararlaştırılması gerekmektedir, şahsi isteklere göre değişiklik
yapıldığında oldukça fazla iletişim sorunu yaşanmaktadır.

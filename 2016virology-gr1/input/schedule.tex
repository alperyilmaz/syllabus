\section{Ders Planı}\label{ders-planux131}

Derslerin beklenenden uzun olması halinde aşağıdaki planda küçük
değişikler olabilir.

\subsubsection{16 Şubat. Giriş ve
Tanıtım}\label{ux15fubat.-giriux15f-ve-tanux131tux131m}

Viroloji ve ilgili alanların tanıtımı. Virüslerdeki karmaşıklığa
(complexity) ve çeşitliliğe genel bakış. Genom, protein açısından
virüslerin sahip olduğu farklılıklar. Kılıflı ve kılıfsız virüsler
arasında temel farklar.

\subsubsection{23 Şubat. Virolojide Kullanılan Metodlar ve Virus
Yapısı}\label{ux15fubat.-virolojide-kullanux131lan-metodlar-ve-virus-yapux131sux131}

Ders kitabının \emph{Chapter 2} ve \emph{Chapter 3} kısımları.

\subsection{Mekanizmlar}\label{mekanizmlar}

\subsubsection{1 Mart. Hücreye Bağlanma, Giriş. Translasyon ve
Taşıma}\label{mart.-huxfccreye-baux11flanma-giriux15f.-translasyon-ve-taux15fux131ma}

Ders kitabının \emph{Chapter 5} ve \emph{Chapter 6} kısımları.

\subsubsection{8 Mart. Virus Genom Replikasyonu, Assembly ve Çıkış,
Virus
Sınıflandırması}\label{mart.-virus-genom-replikasyonu-assembly-ve-uxe7ux131kux131ux15f-virus-sux131nux131flandux131rmasux131}

Ders kitabının \emph{Chapter 7} , \emph{Chapter 8} ve \emph{Chapter 10}
kısımları.

\subsection{Virüs Grupları}\label{viruxfcs-gruplarux131}

\subsubsection{15 Mart. Herpesvirüler ve Diğer dsDNA
Virüsler}\label{mart.-herpesviruxfcler-ve-diux11fer-dsdna-viruxfcsler}

Ders kitabının \emph{Chapter 11} kısmı.

\subsubsection{22 Mart. Parvovirüsler ve Diğer ssDNA
Virüsler}\label{mart.-parvoviruxfcsler-ve-diux11fer-ssdna-viruxfcsler}

Ders kitabının \emph{Chapter 12} kısmı.

\subsubsection{29 Mart. Reovirüsler ve Diğer dsRNA
Virüsler}\label{mart.-reoviruxfcsler-ve-diux11fer-dsrna-viruxfcsler}

Ders kitabının \emph{Chapter 13} kısmı.

\subsubsection{5 Nisan. Picornavirüsler ve Diğer Pozitif ssRNA
Virüsler}\label{nisan.-picornaviruxfcsler-ve-diux11fer-pozitif-ssrna-viruxfcsler}

Ders kitabının \emph{Chapter 14} kısmı.

\subsubsection{(12 Nisan). Vize}\label{nisan.-vize}

\subsubsection{19 Nisan. Rhabdovirüsler ve Diğer Negatif ssRNA
Virüsler}\label{nisan.-rhabdoviruxfcsler-ve-diux11fer-negatif-ssrna-viruxfcsler}

Ders kitabının \emph{Chapter 15} kısmı.

\subsubsection{26 Nisan. Retrovirüsler}\label{nisan.-retroviruxfcsler}

Ders kitabının \emph{Chapter 16} kısmı.

\subsubsection{3 Mayıs. Retrovirüsler (devam) \&
HIV}\label{mayux131s.-retroviruxfcsler-devam-hiv}

Ders kitabının \emph{Chapter 16} ve \emph{Chapter 17} kısımları.

\subsubsection{10 Mayıs. Hepadnavirüsler ve Diğer Reverse-Transcription
DNA
Virüsler}\label{mayux131s.-hepadnaviruxfcsler-ve-diux11fer-reverse-transcription-dna-viruxfcsler}

Ders kitabının \emph{Chapter 18} kısmı.

\subsubsection{17 Mayıs. Virüslerin Gelişimi ve Ortaya
Çıkışları}\label{mayux131s.-viruxfcslerin-geliux15fimi-ve-ortaya-uxe7ux131kux131ux15flarux131}

Ders kitabının \emph{Chapter 20} ve \emph{Chapter 21} kısımları.

\subsubsection{24 Mayıs. İnfektivite Direnci ve Aşılar (Son
Ders)}\label{mayux131s.-infektivite-direnci-ve-aux15fux131lar-son-ders}

Ders kitabının \emph{Chapter 23} ve \emph{Chapter 24} kısımları.
